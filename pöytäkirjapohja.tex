\documentclass[finnish]{vakioasiakirja}
\usepackage[T1]{fontenc}
\usepackage[utf8]{inputenc}

\newcommand{\puheenjohtaja}{Marko Suovula}
\newcommand{\sihteeri}{Vesa-Pekka Palmu}
\newcommand{\pktarkastajaA}{Martti Paalanen}
\newcommand{\pktarkastajaB}{Eero Af Heurlin}
\date{10.1.2017}
\newcommand{\alkuaika}{20.15}
\newcommand{\loppuaika}{20.30}


\newcommand{\namesig}[2][5cm]{%
  \begin{tabular}{@{}p{#1}@{}}
    \\[2\normalbaselineskip] \hrule \medskip #2 \\[0pt]
  \end{tabular}
}

\newcounter{pkPykala}
\setcounter{pkPykala}{1}
\newcommand{\pkP}[1]{
	\section{\arabic{pkPykala} #1}
	 \addtocounter{pkPykala}{1}
}

\author{Helsinki Hacklab Ry}
\address{
  Takkatie 18\\
  00370 Helsinki
}


\type{Pöytäkirja}

\title{Hallituksen kokous 2/2017}

\begin{document}

\maketitle

\section{Läsnä}

\noindent Eero af Heurlin, Teppo Jussmäki, Taneli Kaivola, Martti Paalanen, Vesa-Pekka Palmu ja Marko Suovula, 

\pkP{Kokouksen avaus}

\noindent \puheenjohtaja avasi kokouksen kello \alkuaika

\pkP{Kokouksen järjestäytyminen}

\noindent Valittiin kokouksen puheenjohtajaksi \puheenjohtaja ja sihteeriksi \sihteeri. Pöytäkirjantarkastajiksi valittiin \pktarkastajaA ja \pktarkastajaB.

\pkP{Kokouksen päätösvaltaisuus}

\noindent Kokous on päätösvaltainen, koska paikalla on vähintään puolet hallituksen jäsenistä.

\pkP{Kokouksen työjärjestys}

\noindent Hyväksyttiin esityslista kokouksen työjärjestykseksi

\pkP{Jäsenasiat}

\noindent Hyväksyttiin uudet jäsenet ja avainjäsenet liitteen 1 mukaisesti.

\pkP{Kokouksen päättäminen}

\noindent \puheenjohtaja päätti kokouksen kello \loppuaika.\\A\\B\\C\\D\\E\\F

\vfill

\section{Liitteet} 

\align Liite 1. Hyväksytyt jäsenet

\section{Pöytäkirjan vakuudeksi}
\nopagebreak

\noindent
\namesig{\puheenjohtaja \\puheenjohtaja} \hfill \namesig{\sihteeri \\sihteeri}
\nopagebreak
\linebreak
\vspace{5cm}
\namesig{\pktarkastajaA \\pöytäkirjantarkastaja} \hfill \namesig{\pktarkastajaB \\pöytäkirjantarkastaja}

\end{document}